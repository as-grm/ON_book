\chapter{Uvod}

Začetki sodobne oceanske navigacije segajo v obdobje petnajstega stoletja, ko je odprava Ferdinanda Magelana prvič obplula svet. Na tem potovanju je žal kapitan Magelan tragično izgubil življenje na Filipinih v boju za Mactan (1521). Ladja Victorija se tako vrne brez vodje odprave in prvič v evropski sodobni zgodovini obpluje svet.

To obdobje je obdobje junakov in pionirjev, ki so prvi odkrivali svet s plovbo po širnih oceanih. Pri navigaciji so si pomagali s primitivnimi navigacijskimi pripomočki, kakor so tedaj vedeli in znali. Velikokrat se je zgodilo, da niso vedeli kje so, oziroma se jim je le dozdevalo kje so. To so bili časi navigacije, ko je bilo še veliko v rokah vsemogočnega Pozejdona ali Neptuna in pomočnika za veter Aeolusa. 

Velik napredek v oceanski navigaciji pride s kasnejšimi odpravami Vasco de Gama, Francis Drage in James Cook. Prav izum kronometra, bolj natančno ladijskega kronometra z oznako H-4, ki ga je izumil John Harrison, sekstanta, kot ga danes poznamo, ki ga je izdelal John Hadley in navtičnega almanaha, se je pričelo moderno obdobje oceanske oziroma astronomske navigacije.

Principi sodobne oceanske navigacije temeljijo na sferni trigonometriji. Da bi lahko določili naš položaj na oceanu, ali s pomočjo meritve časa in hitrosti ali s pomočjo meritve višine nebesnih teles, vseskozi potrebuje način preračun s katerim določimo položaj opazovalca. Velik del potrebnih enačb so prav enačbe sferne trigonometrije, ki jih potrebujemo za določitev položaja opazovalca s pomočjo podatkov meritev. Prvo poglavje knjige je zato namenjeno predstavitvi sferne trigonometrije. Prikaže se postopek izpeljave osnovnih izrekov sferne trigonometrije iz osnovnih matematičnih principov. Čeprav v določenih primerih, opis problema s pomočjo sfere ni najbolj natančen, je pa še vedno zelo natančen za uporabo v pomorstvu. S to natančnostjo bomo v večino primerih tudi zadovoljni.

Naslednja poglavja razpredajo o osnovah plovbe po Zemlji kot sferi, kjer poteka razlaga pojmov \emph{loksodromske} in \emph{ortodromske} plovbe. V začetku si pogledamo osnove cilindrične in Merkatorjeve projekcije zemeljske oble na ravnino, ki jo imenujemo pomorska karta. Vsak pomorščak mora znati izdelati tako imenovano \emph{belo karto}, to je Merkatorjevo karto za določeno geografsko področje. Tako sledi opis postopka izdelave Merkatorjeve karte. Poglavje se zaključi z izdelavo plana potovanja. 

V naslednjem poglavju so razložene osnove astronomske navigacije, z opisom vseh klasičnih metod določanja parametrov položaja opazovalca. Najbolj poznana med njimi je \emph{višinska metoda} ali metoda \emph{Marcq de Saint-Hilaire}, imenovana po njenem izumitelju, admiralu  \emph{Marcq de Blond de Saint-Hilaire} \cite{duval1966admiral}. V začetku pričnemo z obravnavo koordinatnih sistemov, ki so temelj izdelave astronomskega navtičnega trikotnika. Nato se posvetimo obravnavi časa in časovnih kotov, ki so posledica gibanja Zemlje. Pri določevanju položaja opazovalca merimo višino nebesnega teles, ki jo je potrebno popraviti zaradi različnih napak. Opišejo se napake in postopki korekcije. Vsi astronomski podatki o nebesnih telesih so združeni v \emph{Efemeridah}, ki so del \emph{Navtičnega almanaha}. Efemeride so lahko v papirnati ali digitalni obliki. Oboji vsebujejo specifične podatke, ki jih navigator potrebuje za določitev astronomskega položaja. Sledijo opisi in prikazi uporabe različnih metod določanja položaja. Poglavje zaključimo z obravnavo vzhoda in zahoda nebesnih teles, ter praktično uporabo pri določanju napake kompasa.

V zadnjem poglavju je opisan pregled uporabe računalnikov, tablic in telefona za uporabo v Oceanski navigaciji. Danes, je večino računskih postopkov digitaliziranih in posplošenih z uporabo naprednih algoritmov za določanje položaja opazovalca. Prikazali bomo uporabo določenih sistemov in razložili ozadje delovanja. Kot podpora k poglavju sodi tudi dopolnilno poglavje, ki vsebuje opis specifičnih komponent računalniških programov. Programi so delo avtorja ali pa pridobljeni iz spleta.  
